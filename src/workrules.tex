\documentclass[10pt,a4paper,uplatex]{jsarticle}
\usepackage{bm}
\usepackage{graphicx}
\usepackage{url}
\usepackage[truedimen,left=25truemm,right=25truemm,top=25truemm,bottom=25truemm]{geometry}
\usepackage{jpdoc}

\def\title{就業規則}

\newcounter{NumOfMembers}\setcounter{NumOfMembers}{1}
\newcounter{DocumentsPerMember}\setcounter{DocumentsPerMember}{1}
\def\Name#1{\ifcase#1\or
株式会社RICOS\or%甲
\else\fi}

\def\Address#1{\ifcase#1\or
東京都千代田区丸の内二丁目3番2号 \or%甲
\else\fi}

\def\Representative#1{\ifcase#1\or
代表取締役 井原 遊 \or%甲
\else\fi}

\def\BlankSignature#1{\ifcase#1\or
0\or%甲
0\else0\fi}

\begin{document}
\newpage
{\centering \Large\bf \title  \vskip 0em}
\vskip 2em
\section{就業規則}
\subsection{総則}
\article{目的}
この就業規則(以下「規則」という。)は、労働基準法(以下「労基法」という。)第89条に基づき、株式会社RICOS(以下、「会社」という。)の労働者の就業に関する事項を定めるものである。

\article{適用範囲}
この規則は、株式会社RICOSの労働者に適用する。

\article{規則の遵守}
会社は、この規則に定める労働条件により、労働者に就業させる義務を負う。また、労働者は、この規則を遵守しなければならない。

\subsection{採用、異動等}
\article{採用手続}
会社は、入社を希望する者の中から選考試験を行い、これに合格した者を採用する。

\article{採用時の提出書類}
労働者として採用された者は、採用された日から2週間以内に次の書類を提出しなければならない。
\begin{enumerate}
    \itm 履歴書
    \itm 住民票の写し又は個人番号カードのコピー
    \itm その他会社が指定するもの
\end{enumerate}
\term
前項の定めにより提出した書類の記載事項に変更を生じたときは会社に変更事項を届け出なければならない。

\article{試用期間}
労働者として新たに採用した者については、採用した日から3か月間を試用期間とする。
\term
前項について、会社が特に認めたときは、この期間を短縮し、又は設けないことがある。
\term
試用期間中に労働者として不適格と認めた者は、解雇することがある。ただし、入社後32日を経過した者については、\ref{para:dismissal}に定める手続によって行う。
\term
試用期間は、勤続年数に通算する。

\article{労働条件の明示}
会社は、労働者を採用するとき、採用時の賃金、就業場所、従事する業務、労働時間、休日、その他の労働条件を記した、労働条件通知書及びこの規則を交付して、又は雇用契約書に記載して労働条件を明示するものとする。

\article{人事異動}
会社は、業務上必要がある場合に、労働者に対して就業する場所及び従事する業務の変更を命ずることがある。
\term
会社は、業務上必要がある場合に、労働者を在籍のまま関係会社へ出向させることがある。
\term
前2項の場合、労働者は正当な理由なくこれを拒むことはできない。

\article{休職}
労働者が、次のいずれかに該当するときは、所定の期間休職とする。
\label{para:layoff}
\begin{enumerate}
    \itm 業務外の傷病による欠勤が2か月を超え、なお療養を継続する必要があるため勤務できないとき。\par 2年以内
    \label{para:sick}
    \itm 前号のほか、特別な事情があり休職させることが適当と認められるとき。\par 必要な期間
\end{enumerate}
\term
休職期間中に休職事由が消滅したときは、原則として元の職務に復帰させる。ただし、元の職務に復帰させることが困難又は不適当な場合には、他の職務に就かせることがある。
\term
\ref{para:layoff}\ref{para:sick}により休職し、休職期間が満了してもなお傷病が治癒せず就業が困難な場合は、休職期間の満了をもって退職とする。

\subsection{服務規律}

\article{服務}
労働者は、職務上の責任を自覚し、誠実に職務を遂行するとともに、会社の指示命令に従い、職務能率の向上及び職場秩序の維持に努めなければならない。

\article{遵守事項}
労働者は、以下の事項を守らなければならない。\label{para:obey}
\begin{enumerate}
    \itm 許可なく職務以外の目的で会社の施設、物品等を使用しないこと。
    \itm 職務に関連して自己の利益を図り、又は他より不当に金品を借用し、若しくは贈与を受ける等不正な行為を行わないこと。
    \itm 勤務中は職務に専念し、正当な理由なく勤務場所を離れないこと。
    \itm 会社の名誉や信用を損なう行為をしないこと。
    \itm 在職中及び退職後においても、業務上知り得た会社、取引先等の機密を漏洩しないこと。
    \itm 酒に酔った状態(アルコールの影響により正常な業務ができないおそれがある状態をいう。)で就業しないこと。
    \itm その他労働者としてふさわしくない行為をしないこと。
\end{enumerate}

\article{セクシュアルハラスメントの禁止}
性的言動により、他の労働者に不利益や不快感を与えたり、就業環境を害するようなことをしてはならない。
\label{para:sexsualharrassment}

\article{職場のパワーハラスメントの禁止}
職務上の地位や人間関係などの職場内の優位性を背景にした、業務の適正な範囲を超える言動により、他の労働者に精神的・身体的な苦痛を与えたり、就業環境を害するようなことをしてはならない。
\label{para:powerharrassment}

\article{個人情報保護及び機密保持義務}
労働者は、会社及び取引先等に関する情報の管理に十分注意を払うとともに、自らの業務に関係のない情報を不当に取得してはならない。
\label{para:confidentiality}
\term
労働者は、会社の業務上知った、もしくは知り得た一切の機密、ノウハウ、データを記録した媒体等、および会社が秘密として指示した事項を秘密として保持しなければならず、会社の承諾なしに、社外に漏洩してはならない
\term
労働者は、職場又は職種を異動あるいは退職するに際して、自らが管理していた会社及び取引先等に関するデータ・情報書類等を速やかに返却しなければならない。

\article{始業及び終業時刻の記録}
労働者は、始業及び終業時に出勤簿に自ら記載し、始業及び終業の時刻を記録しなければならない。

\article{遅刻、早退、欠勤等}
労働者は遅刻、早退若しくは欠勤をし、又は勤務時間中に私用で事業場から外出する際は、事前に会社に対し申し出て承認を得なければならない。ただし、やむを得ない理由で事前に申し出ることができなかった場合は、事後に速やかに届出をし、承認を得なければならない。
\term
傷病のため継続して8日以上欠勤するときは、医師の診断書を提出しなければならない。




\subsection{労働時間、休憩及び休日}

\article{始業・終業時刻}
\label{para:time}
始業及び終業の時刻については、労働者の自主的決定に委ねるものとする。
\term
会議等のため、事前に日程調整を行った上で約した場合は、その時間に勤務しなければならない。

\article{休憩時間}
1日の労働時間が6時間を超える場合は少なくとも45分、8時間を超える場合は少なくとも1時間の休憩時間を付与する。
\term
前項の休憩時間以上を、1日の始業及び終業間の任意の時間に労働者の自主的決定によって休憩しなければならない。
\term
会議等のため、事前に日程調整を行った上で約した場合は、その時間を休憩時間としてはならない。

\article{標準労働時間}
\label{para:standardworktime}
標準となる1日の労働時間は6時間00分とし、週あたり30時間とする。ただし、個別の契約で、週あたりの標準労働時間を定めたときは、その5分の1を1日の標準労働時間とみなす。

\article{清算期間及び総労働時間}
清算期間は1か月間とし、毎月1日を起算日とする。
\term
清算期間中に労働すべき総労働時間は、標準として120時間とする。ただし、個別の契約で、週あたりの標準労働時間を定めたときは、その4倍を清算期間中に労働すべき総労働時間とする。
\term
清算期間中に労働すべき総労働時間は、本部長以上の管理監督者を除き法定労働時間を超えないものとし、会社及び労働者はこれを遵守しなければならない。
\term
一清算期間における勤務時間が、前項の労働すべき総労働時間に満たなかった場合は、基本給のうちその満たない時間に相当する部分の額は支給しないことができる。
\label{para:totalworktime}
\term
清算期間中の休日数は少なくとも4日以上で\ref{para:legal_holiday}3項の法定休日を充足するようにしなければならない。
\term
労働者は、フレキシブルタイムの範囲内で深夜労働をする場合は事前に会社の許可を得なければならない。ただし、個別の契約で固定深夜割増賃金が設定されている場合は除く。

\article{休日}
休日は、次のとおりとする。
\label{para:holiday}
\begin{enumerate}
    \itm 土曜日
    \itm 日曜日
    \itm 国民の祝日(日曜日と重なったときは翌日)
    \itm 年末年始(12月30日~1月4日)
    \itm その他会社が指定する日
\end{enumerate}
\term
業務の都合により会社が必要と認める場合は、あらかじめ前項の休日を他の日と振り替えることがある。
\term
\ref{para:holiday}の規定は、労働者の選択により清算期間中の休日数以上で任意に振り替えることができる。

\article{法定休日}
\label{para:legal_holiday}
労基法第35条の休日として、前条第1項二号の日を指定する。
\term 業務の都合により会社が必要と認める場合又は労働者の選択によって、他の日と振り替えることができる。
\term 前項により振替を行う場合は、令和5年10月1日を起算日とした4週間毎に4日の休日を満たすようにしなければならない。

\subsection{休暇等}
\article{年次有給休暇}
採用日から6か月間継続勤務し、清算期間中の所定総労働時間の8割以上勤務した労働者に対しては、10日の年次有給休暇を与える。その後1年間継続勤務するごとに、当該1年間において所定労働日数の8割以上出勤した労働者に対しては、下の表のとおり勤続期間に応じた日数の年次有給休暇を与える。

\hspace{2cm}\begin{tabular}{|l|p{2.5em}|p{2.5em}|p{2.5em}|p{2.5em}|p{2.5em}|p{2.5em}|p{2.5em}|} \hline
  勤続期間 & 6か月 & 1年6か月& 2年6か月& 3年6か月& 4年6か月& 5年6か月& 6年6か月 \\ \hline \hline
  付与日数 & 10日 & 11日& 12日& 14日& 16日& 18日& 20日 \\ \hline
\end{tabular}

\label{para:paid_vacation}
\term
前項の規定にかかわらず、週所定労働時間30時間未満であり、かつ、週所定労働日数が4日以下(週あたりの標準労働時間を1日の標準労働時間で除したものを週所定労働日数とし、小数点以下を切り上げる。また、週以外の期間によって所定労働日数を定める労働者については年間所定労働日数が216日以下である者をいう。)の労働者に対しては、下の表のとおり所定労働日数及び勤続期間に応じた日数の年次有給休暇を与える。

\hspace{2cm}\begin{tabular}{|p{3em}|p{6.5em}|p{2.5em}|p{2.5em}|p{2.5em}|p{2.5em}|p{2.5em}|p{2.5em}|p{2.5em}|p{2.5em}|} \hline
  週所定労働日数 & 1年間の所定労働日数 & \multicolumn{7}{|c|}{勤続期間} \\ \cline{3-9}
  & & 6か月 & 1年6か月& 2年6か月& 3年6か月& 4年6か月& 5年6か月& 6年6か月 \\ \hline \hline
  4日 & 169日~216日 & 7日 & 8日& 9日& 10日& 12日& 13日& 15日 \\ \hline
  3日 & 121日~168日 & 5日 & 6日& 6日&  8日&  9日& 10日& 11日 \\ \hline
  2日 & 73日~120日  & 3日 & 4日& 4日&  5日&  6日&  6日&  7日 \\ \hline
  1日 & 48日~72日   & 1日 & 2日& 2日&  2日&  3日&  3日&  3日 \\ \hline
\end{tabular}

\label{para:paid_vacation_short}
\term
\ref{para:paid_vacation}又は\ref{para:paid_vacation_short}の年次有給休暇は、労働者があらかじめ請求する時季に取得させる。ただし、労働者が請求した時季に年次有給休暇を取得させることが事業の正常な運営を妨げる場合は、他の時季に取得させることがある。
\term
前項の規定にかかわらず、労働者代表との書面による協定により、各労働者の有する年次有給休暇日数のうち5日を超える部分について、あらかじめ時季を指定して取得させることがある。
\term
\ref{para:paid_vacation}及び\ref{para:paid_vacation_short}の出勤率の算定に当たっては、下記の期間については出勤したものとして取り扱う。
\begin{enumerate}
    \itm 年次有給休暇を取得した期間
    \itm 産前産後の休業期間
    \itm 育児休業、介護休業等育児又は家族介護を行う労働者の福祉に関する法律(平成3年法律第76号。以下「育児・介護休業法」という。)に基づく育児休業及び介護休業した期間
    \itm 業務上の負傷又は疾病により療養のために休業した期間
\end{enumerate}
\term
付与日から1年以内に取得しなかった年次有給休暇は、付与日から2年以内に限り繰り越して取得することができる。
\term
前項について、繰り越された年次有給休暇とその後付与された年次有給休暇のいずれも取得できる場合には、繰り越された年次有給休暇から取得させる。
\term
会社は、毎月の賃金計算締切日における年次有給休暇の残日数を、当該賃金の支払明細書に記載して各労働者に通知する。



\article{産前産後の休業}
6週間(多胎妊娠の場合は14週間)以内に出産予定の女性労働者から請求があったときは、休業させる。
\term
産後8週間を経過していない女性労働者は、就業させない。
\term
前項の規定にかかわらず、産後6週間を経過した女性労働者から請求があった場合は、その者について医師が支障がないと認めた業務に就かせることがある。
\term
産前10週間以内、産後10週間以内の女性労働者の休業請求は優先的に取り扱う。
 

\article{母性健康管理の措置}
妊娠中又は出産後1年を経過しない女性労働者から、所定労働時間内に、母子保健法(昭和40年法律第141号)に基づく保健指導又は健康診査を受けるために申出があったときは、次の範囲内で時間内通院を認める。
\begin{enumerate}
    \itm 産前の場合
    \begin{enumerate}
        \itm 妊娠23週まで\par 4週に1回
        \itm 妊娠24週から35週まで\par 2週に1回
        \itm 妊娠36週から出産まで\par 1週に1回
        \itm ただし、医師又は助産師(以下「医師等」という。)がこれと異なる指示をしたときには、その指示により必要な時間
    \end{enumerate}
    \itm 産後(1年以内)の場合
    \begin{enumerate}
        \itm 医師等の指示により必要な時間
    \end{enumerate}
\end{enumerate}

\term
妊娠中又は出産後1年を経過しない女性労働者から、保健指導又は健康診査に基づき勤務時間等について医師等の指導を受けた旨申出があった場合、次の措置を講ずる。
\begin{enumerate}
    \itm 妊娠中の休憩時間について指導された場合は、適宜休憩時間の延長や休憩の回数を増やす。
    \itm 妊娠中又は出産後の女性労働者が、その症状等に関して指導された場合は、医師等の指導事項を遵守するための作業の軽減や勤務時間の短縮、休業等の措置をとる。
\end{enumerate}

\article{育児時間}
1歳に満たない子を養育する労働者から請求があったときは、休憩時間のほか1日について3回、1回について30分の育児時間を与える。

\article{ウェルネス休暇}
労基法第68条により、生理日の就業が著しく困難な女性労働者から請求があったときは、必要な期間休暇を与える。短期的な病気療養その他の健康上の理由で就業が著しく困難な労働者から請求があったときも、同様とする。

\article{育児・介護休業、子の看護休暇等}
労働者のうち必要のある者は、育児・介護休業法に基づく育児休業、介護休業、子の看護休暇、介護休暇、育児のための所定外労働の免除、育児・介護のための時間外労働及び深夜業の制限並びに所定労働時間の短縮措置等(以下「育児・介護休業等」という。)の適用を受けることができる。
\term
育児休業、介護休業等の取扱いについては、「育児・介護休業等に関する規則」で定める。

\article{裁判員等のための休暇}
労働者が裁判員若しくは補充裁判員となった場合又は裁判員候補者となった場合には、次のとおり休暇を与える。
\begin{enumerate}
    \itm 裁判員又は補充裁判員となった場合 \par 必要な日数
    \itm 裁判員候補者となった場合\par 必要な時間
\end{enumerate}


\subsection{賃金}

\article{賃金の構成}
賃金の構成は、次のとおりとする。
\begin{enumerate}
    \itm 基本給
    \itm 諸手当
    \begin{enumerate}
        \itm 通勤手当
    \end{enumerate}
    \itm 割増賃金
    \begin{enumerate}
        \itm 深夜労働割増賃金
    \end{enumerate}
\end{enumerate}

\article{基本給}
基本給は、本人の職務内容、技能、勤務成績等を考慮して各人別に決定する。

\article{休暇等の賃金}
年次有給休暇を取得した日については、標準となる1日の労働時間を勤務したものとみなす。
\term
産前産後の休業期間、育児時間、生理休暇、母性健康管理のための休暇、育児・介護休業法に基づく育児休業期間、介護休業期間、育児休業に準ずる措置及び子の看護休暇期間、裁判員等のための休暇の期間は、清算期間内に労働すべき総労働時間のうち、日割りで提供義務を免じ、相当する賃金の支給を行わない。 
\term 
ウェルネス休暇を取得したときは、前項と同様とする。ただし、直近の年次有給休暇が付与された日からウェルネス休暇を取得した日までに5日以上の年次有給休暇を取得している場合において、清算期間内に一回に限り第一項と同様する。
\term
前項の規定は、清算期間内に労働すべき総労働時間までの労務の提供を妨げるものではない。
\term
\ref{para:layoff}に定める休職期間中は、原則として賃金を支給しない。

\article{臨時休業の賃金}
会社側の都合により、清算期間内の暦日において過半の日において労務の提供が不可能な状態となり、労基法第12条に規定する平均賃金の6割に満たすことができなくなった場合、その額以上に相当する時間以上の労務があったものとみなして賃金を支給する。

\article{欠勤等の扱い}
欠勤、遅刻、早退及び私用外出については、労務の提供がなかったものとして取り扱う。

\article{賃金の計算期間及び支払日}
賃金は、毎月末日に締め切って計算し、翌月25日に支払う。ただし、支払日が休日に当たる場合は、その前日に繰り上げて支払う。
\term
前項の計算期間の中途で採用された労働者又は退職した労働者については、月額の賃金は当該計算期間の暦日において日割計算して支払う。

\article{賃金の支払と控除}
賃金は、労働者に対し、通貨で直接その全額を支払う。
\term
前項について、労働者が同意した場合は、労働者本人の指定する金融機関の預貯金口座へ振込により賃金を支払う。
\term
次に掲げるものは、賃金から控除する。
\begin{enumerate}
    \itm 源泉所得税
    \itm 住民税
    \itm 健康保険、厚生年金保険及び雇用保険の保険料の被保険者負担分
    \itm 労働者代表との書面による協定により賃金から控除することとした社宅入居料、財形貯蓄の積立金及び組合費
\end{enumerate}

\article{賃金の非常時払い}
労働者又はその収入によって生計を維持する者が、次のいずれかの場合に該当し、そのために労働者から請求があったときは、賃金支払日前であっても、既往の労働に対する賃金を支払う。
\begin{enumerate}
    \itm やむを得ない事由によって1週間以上帰郷する場合
    \itm 結婚又は死亡の場合
    \itm 出産、疾病又は災害の場合
    \itm 退職又は解雇により離職した場合
\end{enumerate}

\article{給与改定}
給与改定(昇給及び降給をいう)は、従業員の勤務成績等を総合的に考慮して、毎年4月1日、7月1日、10月1日、1月1日をもって行うものとする。
\term
顕著な業績が認められた労働者については、前項の規定にかかわらず昇給を行うことがある。
\term
給与改定額は、労働者の勤務成績等を考慮して各人ごとに決定する。


\subsection{退職及び解雇}

\article{退職}
労働者が次のいずれかに該当するときは、退職とする。
\begin{enumerate}
    \itm 退職を願い出て会社が承認したとき、又は退職願を提出して14日を経過したとき。
    \itm 期間を定めて雇用されている場合、その期間を満了したとき。
    \itm \ref{para:layoff}に定める休職期間が満了し、なお休職事由が消滅しないとき。
    \itm 死亡したとき。
    \itm 届出又は連絡なく欠勤を続け、その欠勤期間が30日を超え、なお所在が不明のとき。
\end{enumerate}
\term
労働者が退職し、又は解雇された場合、その請求に基づき、使用期間、業務の種類、地位、賃金又は退職の事由を記載した証明書を遅滞なく交付する。

\article{解雇}
労働者が次のいずれかに該当するときは、解雇することがある。
\label{para:dismissal}
\begin{enumerate}
    \itm 勤務状況が著しく不良で、改善の見込みがなく、労働者としての職責を果たし得ないとき。
    \itm 勤務成績又は業務能率が著しく不良で、向上の見込みがなく、他の職務にも転換できない等就業に適さないとき。
    \itm 業務上の負傷又は疾病による療養の開始後3年を経過しても当該負傷又は疾病が治らない場合であって、労働者が傷病補償年金を受けているとき又は受けることとなったとき(会社が打ち切り補償を支払ったときを含む。)。
    \itm 精神又は身体の障害により業務に耐えられないとき。
    \itm 試用期間における作業能率又は勤務態度が著しく不良で、労働者として不適格であると認められたとき。
    \itm \ref{para:disciplinary}に定める懲戒解雇事由に該当する事実が認められたとき。
    \itm 事業の運営上又は天災事変その他これに準ずるやむを得ない事由により、事業の縮小又は部門の閉鎖等を行う必要が生じ、かつ他の職務への転換が困難なとき。
    \itm その他前各号に準ずるやむを得ない事由があったとき。
\end{enumerate}
\term
前項の規定により労働者を解雇する場合は、少なくとも30日前に予告をする。予告しないときは、平均賃金の30日分以上の手当を解雇予告手当として支払う。ただし、予告の日数については、解雇予告手当を支払った日数だけ短縮することができる。
\term
前項の規定は、労働基準監督署長の認定を受けて労働者を\ref{para:disciplinary}に定める懲戒解雇する場合又は次の各号のいずれかに該当する労働者を解雇する場合は適用しない。
\begin{enumerate}
    \itm 日々雇い入れられる労働者(ただし、1か月を超えて引き続き使用されるに至った者を除く。)
    \itm 2か月以内の期間を定めて使用する労働者(ただし、その期間を超えて引き続き使用されるに至った者を除く。)
    \itm 試用期間中の労働者(ただし、14日を超えて引き続き使用されるに至った者を除く。)
\end{enumerate}
\term
\ref{para:dismissal}の規定による労働者の解雇に際して労働者から請求のあった場合は、解雇の理由を記載した証明書を交付する。

\article{退職金}
労働者が退職したとき、退職手当規程に定めるところにより、退職金を支給する。
\label{para:severance}

\subsection{安全衛生及び災害補償}

\article{遵守事項}
会社は、労働者の安全衛生の確保及び改善を図り、快適な職場の形成のために必要な措置を講ずる。
\term
労働者は、安全衛生に関する法令及び会社の指示を守り、会社と協力して労働災害の防止に努めなければならない。
\term
労働者は安全衛生の確保のため、特に下記の事項を遵守しなければならない。
\begin{enumerate}
    \itm 機械設備、工具等の就業前点検を徹底すること。また、異常を認めたときは、速やかに会社に報告し、指示に従うこと。
    \itm 安全装置を取り外したり、その効力を失わせるようなことはしないこと。
    \itm 保護具の着用が必要な作業については、必ず着用すること。
    \itm 喫煙は、所定の場所以外では行わないこと。
    \itm 立入禁止又は通行禁止区域には立ち入らないこと。
    \itm 常に整理整頓に努め、通路、避難口又は消火設備のある所に物品を置かないこと。
    \itm 火災等非常災害の発生を発見したときは、直ちに臨機の措置をとり、安全衛生を担当する者に報告し、その指示に従うこと。
\end{enumerate}

\article{健康診断}
労働者に対しては、採用の際及び毎年1回(深夜労働に従事する者は6か月ごとに1回)、定期に健康診断を行う。
\label{checkup}
\term
 前項の健康診断のほか、法令で定められた有害業務に従事する労働者に対しては、特別の項目についての健康診断を行う。
\label{checkupforhazardous}
\term
長時間の労働により疲労の蓄積が認められる労働者に対し、その者の申出により医師による面接指導を行う。
\term
\ref{checkup}及び\ref{checkupforhazardous}の健康診断並びに前項の面接指導の結果必要と認めるときは、一定期間の就業禁止、労働時間の短縮、配置転換その他健康保持上必要な措置を命ずることがある。

\article{健康管理上の個人情報の取扱い}
会社への提出書類及び身上その他の個人情報(家族状況も含む)並びに健康診断書その他の健康情報は、次の目的のために利用する。
\begin{enumerate}
    \itm 会社の労務管理、賃金管理、健康管理
    \itm 出向、転籍等のための人事管理
\end{enumerate}
\term
労働者の定期健康診断の結果、労働者から提出された診断書、産業医等からの意見書、過重労働対策による面接指導結果その他労働者の健康管理に関する情報は、労働者の健康管理のために利用するとともに、必要な場合には産業医等に診断、意見聴取のために提供するものとする。

\article{安全衛生教育}
労働者に対し、雇入れの際及び配置換え等により作業内容を変更した場合、その従事する業務に必要な安全及び衛生に関する教育を行う。
\term
労働者は、安全衛生教育を受けた事項を遵守しなければならない。
\article{災害補償}
労働者が業務上の事由又は通勤により負傷し、疾病にかかり、又は死亡した場合は、労基法及び労働者災害補償保険法(昭和22年法律第50号)に定めるところにより災害補償を行う。

\subsection{職業訓練}

\article{教育訓練}
会社は、業務に必要な知識、技能を高め、資質の向上を図るため、労働者に対し、必要な教育訓練を行う。
\term
労働者は、会社から教育訓練を受講するよう指示された場合には、特段の事由がない限り教育訓練を受けなければならない。
\term
前項の指示は、教育訓練開始日の少なくとも2週間前までに該当労働者に対し通知する。

\subsection{表彰及び制裁}

\article{表彰}
会社は、労働者が次のいずれかに該当するときは、表彰することがある。
\begin{enumerate}
    \itm 業務上有益な発明、考案を行い、会社の業績に貢献したとき。
    \itm 永年にわたって誠実に勤務し、その成績が優秀で他の模範となるとき。
    \itm 永年にわたり無事故で継続勤務したとき。
    \itm 社会的功績があり、会社及び労働者の名誉となったとき。
    \itm 前各号に準ずる善行又は功労のあったとき。
\end{enumerate}
\term
表彰は、原則として会社の創立記念日に行う。また、賞状のほか賞金を授与する。

\article{懲戒の種類}
会社は、労働者が次条のいずれかに該当する場合は、その情状に応じ、次の区分により懲戒を行うことがある。
\begin{enumerate}
    \itm 譴責\par 始末書を提出させて将来を戒める。
    \itm 減給\par 始末書を提出させて減給する。ただし、減給は1回の額が平均賃金の1日分の5割を超えることはなく、また、総額が1賃金支払期における賃金総額の1割を超えることはない。
    \itm 出勤停止\par 始末書を提出させるほか、14日間を限度として出勤を停止し、その間の賃金は支給しない。
    \itm 懲戒解雇\par 予告期間を設けることなく即時に解雇する。この場合において、所轄の労働基準監督署長の認定を受けたときは、解雇予告手当(平均賃金の30日分)を支給しない。
\end{enumerate}

\article{懲戒の事由}
労働者が次のいずれかに該当するときは、情状に応じ、譴責、減給又は出勤停止とする。
\label{para:disciplinary}
\begin{enumerate}
    \itm 正当な理由なく、しばしば清算期間中の総労働時間の勤務を行わないとき。
    \itm 正当な理由なく、事前に日程調整し了承の上で約した会議等にしばしば出席せず又は遅刻するとき。
    \itm 過失により会社に損害を与えたとき。
    \itm 素行不良で社内の秩序及び風紀を乱したとき。
    \itm 性的な言動により、他の労働者に不快な思いをさせ、又は職場の環境を悪くしたとき。
    \itm 性的な関心を示し、又は性的な行為をしかけることにより、他の労働者の業務に支障を与えたとき。
    \itm \ref{para:obey}、\ref{para:sexsualharrassment}、\ref{para:powerharrassment}又は\ref{para:confidentiality}に違反したとき。
    \itm その他この規則に違反し又は前各号に準ずる不都合な行為があったとき。
\end{enumerate}
\term
労働者が次のいずれかに該当するときは、懲戒解雇とする。ただし、平素の服務態度その他情状によって、\ref{para:dismissal}に定める普通解雇、\ref{para:disciplinary}に定める減給、出勤停止とすることができる。
\begin{enumerate}
    \itm 重要な経歴を詐称して雇用されたとき。
    \itm 正当な理由なく、無断欠勤が32日以上に及び、出勤の督促に応じなかったとき。
    \itm 正当な理由なく、無断で欠席等を繰り返し、8回にわたって注意を受けても改めなかったとき。
    \itm 正当な理由なく、しばしば業務上の指示・命令に従わなかったとき。
    \itm 故意又は重大な過失により会社に重大な損害を与えたとき。
    \itm 会社内において刑法その他刑罰法規の各規定に違反する行為を行い、その犯罪事実が明らかとなったとき(当該行為が軽微な違反である場合を除く。)。
    \itm 素行不良で著しく社内の秩序又は風紀を乱したとき。
    \itm 数回にわたり懲戒を受けたにもかかわらず、なお、勤務態度等に関し、改善の見込みがないとき。
    \itm 職責を利用して交際を強要し、又は性的な関係を強要したとき。
    \itm \ref{para:sexsualharrassment}、\ref{para:powerharrassment}に違反し、その情状が悪質と認められるとき。
    \itm 故意に\ref{para:confidentiality}に違反したとき。
    \itm 許可なく職務以外の目的で会社の施設、物品等を使用したとき。
    \itm 職務上の地位を利用して私利を図り、又は取引先等より不当な金品を受け、若しくは求め若しくは供応を受けたとき。
    \itm 私生活上の非違行為や会社に対する正当な理由のない誹謗中傷等であって、会社の名誉信用を損ない、業務に重大な悪影響を及ぼす行為をしたとき。
    \itm 正当な理由なく会社の業務上重要な秘密を外部に漏洩して会社に損害を与え、又は業務の正常な運営を阻害したとき。
    \itm その他前各号に準ずる不適切な行為があったとき。
\end{enumerate}

\subsection{公益通報者保護}

\article{公益通報者の保護}
会社は、労働者から組織的又は個人的な法令違反行為等に関する相談又は通報があった場合には、別に定めるところにより処理を行う。

\subsection*{附則}
\article{施行日}
本編の規則は、令和3年12月15日から施行する。



\clearpage
\section{育児・介護休業等に関する規則}

\article{育児休業}
育児のために休業することを希望する従業員(日雇従業員を除く)であって、1歳に満たない子と同居し、養育する者は、申出により、育児休業をすることができる。ただし、有期契約従業員にあっては、申出時点において、次のいずれにも該当する者に限り、育児休業をすることができる。
\label{para:ChildcareLayoff}
\begin{enumerate}
    \itm 入社1年以上であること。
    \itm 子が1歳6か月に達する日までに労働契約期間が満了し、更新されないことが明らかでないこと。
\end{enumerate}
\term 前項にかかわらず、次の従業員からの休業の申出は拒むことができる。
\begin{enumerate}
    \itm 申出の日から1年以内(4項の申出をする場合は、6か月以内)に雇用関係が終了することが明らかな従業員
    \itm 1週間の所定労働日数が2日以下の従業員
\end{enumerate}
\term 配偶者が従業員と同じ日から又は従業員より先に育児休業をしている場合、従業員は、子が1歳2か月に達するまでの間で、出生日以後の産前・産後休業期間と育児休業期間との合計が1年を限度として、育児休業をすることができる。
\label{para:ikukyuplus}
\term 次のいずれにも該当する従業員は、子が1歳6か月に達するまでの間で必要な日数について育児休業をすることができる。なお、育児休業を開始しようとする日は、原則として子の1歳の誕生日に限るものとする。
\begin{enumerate}
    \itm 従業員又は配偶者が原則として子の1歳の誕生日の前日に育児休業をしていること。
    \itm 次のいずれかの事情があること。
    \begin{enumerate}
        \itm 保育所等に入所を希望しているが、入所できない場合
        \itm 従業員の配偶者であって育児休業の対象となる子の親であり、1歳以降育児に当たる予定であった者が、死亡、負傷、疾病等の事情により子を養育することが困難になった場合
    \end{enumerate}
\end{enumerate}
\term 育児休業をすることを希望する従業員は、原則として、育児休業を開始しようとする日の1か月前(\ref{para:ikukyuplus}に基づく1歳を超える休業の場合は、2週間前)までに、育児休業申出書を人事担当者に提出することにより申し出るものとする。
\term 育児休業申出書が提出されたときは、会社は速やかに当該育児休業申出書を提出した者に対し、育児休業取扱通知書を交付する。

\article{介護休業}
要介護状態にある家族を介護する従業員(日雇従業員を除く)は、申出により、介護を必要とする家族1人につき、のべ93日間までの範囲内で3回を上限として介護休業をすることができる。ただし、有期契約従業員にあっては、申出時点において、次のいずれにも該当する者に限り、介護休業をすることができる。
\label{para:NursingLayoff}
\begin{enumerate}
    \itm 入社1年以上であること。
    \itm 介護休業開始予定日から93日を経過する日から6か月を経過する日までに労働契約期間が満了し、更新されないことが明らかでないこと。
\end{enumerate}
\term 前項にかかわらず、労使協定により除外された次の従業員からの休業の申出は拒むことができる。
\begin{enumerate}
    \itm 入社1年未満の従業員
    \itm 申出の日から93日以内に雇用関係が終了することが明らかな従業員
    \itm 1週間の所定労働日数が2日以下の従業員
\end{enumerate}
\term 要介護状態にある家族とは、負傷、疾病又は身体上若しくは精神上の障害により、2週間以上の期間にわたり常時介護を必要とする状態にある次の者をいう。
\begin{enumerate}
    \itm 配偶者
    \itm 父母
    \itm 子
    \itm 配偶者の父母
    \itm 祖父母
    \itm 兄弟姉妹
    \itm 孫
\end{enumerate}
\term 介護休業をすることを希望する従業員は、原則として、介護休業を開始しようとする2週間前までに、介護休業申出書を人事担当者に提出することにより申し出るものとする。
\term 介護休業申出書が提出されたときは、会社は速やかに当該介護休業申出書を提出した者に対し、介護休業取扱通知書を交付する。

\article{子の看護休暇}
小学校就学の始期に達するまでの子を養育する従業員(日雇従業員を除く)は、負傷し、又は疾病にかかった当該子の世話をするために、又は当該子に予防接種や健康診断を受けさせるために、就業規則に規定する年次有給休暇とは別に、当該子が1人の場合は1年間につき5日、2人以上の場合は1年間につき10日を限度として、子の看護休暇を取得することができる。この場合の1年間とは、4月1日から翌年3月31日までの期間とする。ただし、次の従業員からの申出は拒むことができる。
\label{para:ChildNursingAbsence}
\begin{enumerate}
    \itm 入社6か月未満の従業員
    \itm 1週間の所定労働日数が2日以下の従業員
\end{enumerate}
\term 子の看護休暇は、半日単位で取得することができる。

\article{育児休業に準ずる休業措置}
育児のために休業することを希望する従業員であって、\ref{para:ChildcareLayoff}に定める育児休業に該当せず、3歳に満たない子と同居し、養育する者は、育児休業に準ずる休業の申出をすることができる。
\term
前項に定める育児休業に準ずる休業の申出期間中は、時間単位、日単位、月単位等で育児に際して必要な時間について、休業することができ、フレックタイム制適用従業員については、標準労働時間から控除することができる。
\term
前項に定める休業期間の給与は支給しないことができ、賞与の算定から控除することができる。

\article{介護休暇}
要介護状態にある家族の介護その他の世話をする従業員(日雇従業員を除く)は、就業規則に規定する年次有給休暇とは別に、対象家族が1人の場合は1年間につき5日、2人以上の場合は1年間につき10日を限度として、介護休暇を取得することができる。この場合の1年間とは、4月1日から翌年3月31日までの期間とする。ただし、次の従業員からの申出は拒むことができる。
\label{para:NursingAbsence}
\begin{enumerate}
    \itm 入社6か月未満の従業員
    \itm 1週間の所定労働日数が2日以下の従業員
\end{enumerate}
\term 介護休暇は、半日単位で取得することができる。

\article{給与等の取扱い}
基本給その他の月毎に支払われる給与の取扱いは次のとおりとする。
\begin{enumerate}
    \itm 育児休業(\ref{para:ChildcareLayoff})及び介護休業(\ref{para:NursingLayoff})の適用をした期間については、原則として支給しない。
    \itm \ref{para:NursingAbsence}及び\ref{para:ChildNursingAbsence}の制度の適用を受けた日又は時間については、原則として無給とする。
\end{enumerate}
\term 賞与については、その算定対象期間に育児・介護休業をした期間が含まれる場合には、出勤日数により日割りで計算した額とすることができる。ただし、全額の支給を妨げない。
\term 退職金の算定に当たっては、育児・介護休業をした期間は勤務したものとして勤続年数を計算するものとする。
\term 年次有給休暇の権利発生のための出勤率の算定に当たっては、育児・介護休業をした日は出勤したものとみなす。

\article{育児休業等に関するハラスメントの防止}
すべての従業員は本規則に定める制度の申出・利用に関して、当該申出・利用する従業員の就業環境を害する言動を行ってはならない。

\article{法令との関係}
育児・介護休業、子の看護休暇、介護休暇、育児・介護のための所定外労働の制限、時間外労働及び深夜業の制限、育児短時間勤務並びに介護短時間勤務に関して、この規則に定めのないことについては、育児・介護休業法その他の法令の定めるところによる。


\clearpage
\section{退職手当規程}

\article{目的}
この規程は、就業規則\ref{para:severance}に基づき従業員の退職金について定めたものである。

\article{適用範囲}
この規程による退職金制度は、会社に雇用され勤務するものすべてに適用する。

\article{退職金の算定方法}
退職金は別表1で定めるところにより、退職時における最終賃金月における課税支給額に支給率を乗じたものとする。
\term 前項の支給率の算定にあたって、勤続年数は、勤続日数を365.2425で除してするものとする。
\term 1円未満の端数については切り捨てるものとする。

\article{特別功労金}
在職中、特に功労があったと認められる従業員に対して、退職金に特別功労金を加算して支給することがある。支給額は、その都度、その功労の程度を勘案して、退職金に0を超え、50未満の特別功労金乗率を乗じたものを加算するものとする。

\article{支払の時期および方法}
退職金は、退職または解雇の日から60日以内に通貨で、支給対象者の指定する金融機関口座への振込みにより支払う。



\clearpage
\subsection*{別表1 基本退職金支給率表}

\begin{table}[!!htb]
  \begin{tabular}{|l|c|} \hline
    勤続年数(A) & 最終賃金月の課税支給額に対する乗率 \\ \hline \hline
    1年以下          &  $\frac{\lfloor 5A \rfloor-\lfloor \pi A \rfloor+\lfloor 7A \rfloor}{\mathrm{e}^\pi}$  \\
    1年を超え、5年以下 & $\frac{1}{4}\sin(A\pi^4)+\frac{1}{\pi}\cos(A\pi^3)+\frac{A}{\pi}+0.2$ \\
    5年を超え、10年以下 & $\Gamma(\sqrt{A})$ \\
    10年超           & $\frac{A}{17}\cos(13A)+\frac{A}{13}\sin(11A)+\log_{10} \Gamma(\frac{A}{2})$ \\ \hline
  \end{tabular}
\end{table}


\clearpage
\section{専門業務型裁量労働制規程}

\article{目的}
この規程は、労使協定に基づき、専門業務型裁量労働制の対象となる従業員について定めたものである。

\article{適用範囲}
\label{para:discretiontarget}専門業務型裁量労働制は、以下の各号の業務に従事する従業員のうち適用することについて、会社と合意した従業員(以下、「裁量労働適用者」という。)に適用する。
\begin{enumerate}
    \itm 研究部門における、新商品若しくは新技術の研究開発又は人文科学若しくは自然科学に関する研究の業務
    \itm 開発部門における、情報処理システムの分析又は設計の業務
    \itm 開発部門及び営業部門における、事業運営において情報処理システムを活用するための問題点の把握又はそれを活用するための方法に関する考案若しくは助言の業務
\end{enumerate}
\term 前項の規定に関わらず、会社が裁量労働適用者に専門業務型裁量労働制を適用することが適当でないと認めた場合には、会社は当該従業員に専門業務型裁量労働制を適用しないことができる。

\article{裁量労働}
裁量労働適用者に対しては、会社は、業務遂行の手段及び時間配分の決定等について具体的な指示をせず、裁量労働適用者の裁量に委ねるものとする。

\article{始業・終業時刻及び休憩時間}
裁量労働適用者の始業・終業時刻及び休憩時間は、始業時間10時、終業時間17時及び休憩時間1時間30分を基本とするが、業務遂行上の必要による変更は、裁量労働適用者の裁量に委ねるものとする。

\article{みなし労働時間}
\label{para:discretiontargetworktime}
\ref{para:discretiontarget}に定める裁量労働適用者が対象業務に従事した日は、次の各号に定める時間を労働したものとみなす。
\begin{enumerate}
    \itm 5時間30分
\end{enumerate}

\article{休日}
休日は、原則として次のとおりとする。
\begin{enumerate}
    \itm 土曜日及び日曜日
    \itm 国民の祝日(日曜日と重なったときは翌日)
    \itm 年末年始(12月30日〜1月4日)
    \itm その他会社が指定する日
\end{enumerate}
\term 前項にかかわらず、裁量労働適用者の裁量で、業務遂行に必要と認めるときは、前項の休日を休日としないことが出来る。
\term 前項の規定は、労基法第35条に定める休日として、4月1日を起算日として、少なくとも4週あたり4日以上としなければならない。

\article{労働日及び1月あたりの労働日数}
労働日は、裁量労働適用者が、会社のために業務を行った日をいう。
\term 裁量労働適用者の裁量で、業務遂行が十分で、業務が不要と認める日は、前条に定める休日以外の日を休日とすることが出来る。
\term 1月あたりの労働日数は、第1項の労働日及び有給休暇取得日を、1日から末日まで合計したものとする。

\article{最低労働日数と、欠勤に関する特別の取扱}
裁量労働適用者の1月あたりの労働日数が16日以上であるときは、雇用契約に定める1週あたりの標準労働時間を1月に換算したものに足りないときも、会社は欠勤があったものとみなさない。
\term 有給休暇は、前項の労働日数に含めるものとする。
\term 裁量労働適用者は、業務外の傷病、その他の特別な事情により、前項の労働日数を満たせないときは、その解消に必要な期間、休職をしなければならない。

\article{賃金の取扱}
会社は、裁量労働制が高度な専門業務を有する裁量労働適用者の職務遂行能力とその成果に対して賃金を支払う趣旨に鑑み、前条に定める著しい欠勤がない及び法律による割増賃金が発生しない限り、労働日数によらず、賃金は雇用契約に定める金額を支払う。
\term 会社は、裁量労働適用者を、労働時間による評価ではなく、成果に対する評価を行わなければならず、その成果に対して給与改定しなければならない。

\article{深夜労働}
裁量労働制適用者が深夜労働をする場合は、事前に会社の許可を得なければならない。

\article{健康と福祉の確保}
会社は、裁量労働適用者の健康と福祉を確保するために、次の措置を講ずる。
\begin{enumerate}
    \itm 会社は、裁量労働適用者の健康状態を把握するために、開発にかかるログのタイムスタンプ、コミュニケーションツールのタイムスタンプ及び電子メールのタイムスタンプ等の記録により、裁量労働適用者の業務従事状況を把握する。
    \itm 会社は、必要に応じて、裁量労働適用者に健康状態等について聴取を行い、労働日数の調整を推奨する。
    \itm 精神及び身体の健康状態についての相談室を設置する。
\end{enumerate}

\article{苦情処理}
会社は、裁量労働適用者から苦情があった場合には、次の手続を行うものとする。
\begin{enumerate}
    \itm 相談場所: 個別に調整する
    \itm 担当者: 代表取締役
    \itm 開設日時: 随時
    \itm 取り扱う苦情の範囲: 専門業務型裁量労働制の運用に関する全般及び裁量労働適用者の評価制度及び賃金制度等の処遇全般
    \itm 会社は、相談者の秘密を厳守する。
\end{enumerate}

\article{裁量労働に関する同意及びその撤回}
会社は、専門業務型裁量労働制の適用を受けることになる者に対して、あらかじめ雇用契約書その他の合意があったことを示す書面又は電磁的記録によって、明示的に専門業務型裁量労働制を適用することについての同意を得るものとする。この同意を得るにあたっては、会社は、制度概要等を説明し、同意した場合に適用される評価制度及び賃金制度の内容並びに同意しなかった場合の配置及び処遇を説明するものとする。
\term 裁量労働適用者は、就業規則に定める賃金の計算期間(賃金月)を単位として、専門業務型裁量労働制の適用に関する同意を撤回することができる。
\term 会社は、第1項の同意をしなかった者又は前項の撤回を行った裁量労働制適用者に対し、各々第1項の同意をしなかったこと又は前項の撤回を行ったことを理由として、処遇等で、本人に不利益な取扱いをしてはならないものとする。

\article{同意の撤回の手続き}
裁量労働適用者が、裁量労働適用者が、前条第2項に定める撤回をしようとするときは、専門業務型裁量労働制の適用を受けることをやめようとする賃金月の初日の14日前までに、書面又は電磁的記録によって会社に通知しなければならない。

\article{記録の保存}
会社は、裁量労働適用者の労働状況、健康・福祉のために講じた措置、苦情処理について講じた措置の記録、同意及び同意の撤回の労働者ごとの記録を、本協定有効期間中及び有効期間満了後3年間保存する。

\clearpage
\section{テレワーク規程}

\article{目的}
この規程は、従業員がテレワークを行う場合において必要な事項について定めたものである。

\article{定義}
在宅勤務とは、労働者の自宅、その他自宅に準じる場所において情報通信機器を利用して業務を行うことをいう。
\term サテライトオフィス勤務とは、会社保有の所属事業場以外の会社専用施設または、会社が指定している他会社所有の共用施設において情報通信機器を利用した業務をいう。
\term モバイル勤務とは、在宅勤務及びサテライトオフィス勤務以外で、かつ、社外で情報通信機器を利用して業務を行うことをいう。
\term 在宅勤務、サテライトオフィス勤務及びモバイル勤務を総称して、テレワークという。

\article{テレワークの申出及び許可}
テレワークを希望する者は、執務環境及びセキュリティ環境を整え、テレワークを申し出るものとする。
\term 前項の申出が業務に支障がないと認められる場合は、会社はテレワークの許可をしなければならない。
\term 従前より労働者の裁量によって事業所以外での勤務を認めている者は、この規程におけるテレワークの許可があったものとみなす。

\article{事業所での就業}
テレワーク適用者は、自身の判断によって、事業所での就業をすることができる。

\article{始業・終業時刻及び休憩時間}
テレワーク適用者は、出勤管理システムにより始業・終業時刻及び休憩時間を記録しなければならない。

\article{情報の取扱}
テレワーク適用者は、第三者が会社の情報を閲覧、複製等しないように最大限の注意をしなければならない。

\article{費用の負担}
テレワーク適用者は、情報通信機器や事務用品等の業務に必要な物品の貸与又は支給を請求することができ、会社が貸与又は支給した物品を使用する場合の費用(通信料等も含む)は会社の負担とする。
\term 貸与及び支給した物品を使用しない場合の費用、水道光熱費その他の費用については、テレワーク適用者の負担とする。

\clearpage
\section{職務発明規程}

\article{目的}
本規程は、株式会社RICOS(以下「会社」という。)に勤務する従業者等が創作した発明等の取扱いについて定め、その発明者としての権利を保障し、発明等の奨励及び発明意欲の向上を図り、社業の成長発展と事業活動を通じた社会貢献の促進に寄与することを目的とする。

\article{定義}本規程において、次の各号に掲げる用語の意義は、それぞれ当該各号に定めるところによる。
\term 「発明等」とは、次に掲げるものをいう。
\begin{enumerate}
    \item 特許法(昭和34年法律第121号)に規定する発明
    \item 実用新案法(昭和34年法律第123号)に規定する考案
    \item 意匠法(昭和34年法律第125号)に規定する意匠
\end{enumerate}
\term 「職務発明等」とは、従業者等がなした発明等であって、その性質上会社の業務範囲に属し、その発明等をするに至った行為が、会社における従業者等の現在又は過去の職務に属するものをいう。
\term 「発明者」とは、発明等を行った従業者等をいう。
\term 「特許等」とは、特許、実用新案登録及び意匠登録の総称をいう。
\term 「特許権等」とは、特許法に規定する特許権、実用新案法に規定する実用新案権及び意匠法に規定する意匠権をいう。
\term 「特許等を受ける権利」とは、特許法に規定する特許を受ける権利、実用新案法に規定する実用新案登録を受ける権利及び意匠法に規定する意匠登録を受ける権利をいう。
\term 特許権等及び特許等を受ける権利には、各々これらに対応する外国の権利を含むものとする。
\term 「出願」とは、特許出願、登録出願等、発明等に関し法令で定められた権利保護のために必要な所定の手続を行うこと(これに対応する外国への出願を含む。)をいう。
\term 「従業者等」とは、会社の役員、従業員(労働契約法に基づいて労働契約を会社と締結した労働者をいう。)又は派遣労働者(労働者派遣事業の適正な運営の確保及び派遣労働者の保護等に関する法律に基づく派遣労働者をいう。)をいう。

\article{届出}
会社の業務範囲に属する発明等を行った従業者等は、遅滞なく会社指定の様式にて「発明届」を作成し、会社に届け出なければならない。
\label{para:invention_notice}
\term 前項の発明等が二人以上の者によって共同でなされたもの(以下「共同発明等」という。)であるときは、前項の発明届を連名で作成するとともに、各発明者の寄与率を記入するものとする。

\article{権利の帰属}
従業者等の職務発明等については、その発明等の完成と同時に、意思表示その他何らの手続を要することなく、会社が特許等を受ける権利を取得する。
\term 会社は、前項の規定により職務発明等について特許等を受ける権利を取得した後、当該特許等を受ける権利又はこれに基づく特許権等を必要とせず、かつ、秘密として管理しないと判断した場合には、当該特許等を受ける権利又はこれに基づく特許権等について無償かつ無制限、無期限の再実施許諾可能な通常実施権(仮通常実施権を含む。)を留保したうえで、これを当該職務発明等を行った従業者等に譲渡することができる。
\term 従業者等が会社の職務上作成したプログラム等(著作権法第2条第1項第10号の2のプログラムの著作物又は同項第10号の3のデータベースの著作物をいう。)その他著作物の著作者は会社とする。また、当該ブログラム等の著作権(著作権法第21条から第28条までの著作権をいい、外国におけるこれらに相当するものを含む。)その他著作権並びに従業者等が会社の職務上作成した商標の商標権(商標登録を受ける権利を含み、外国におけるこれに相当するものを含む。)、回路利用の回路利用配置権(回路配置利用権の設定登録を受ける権利を含み、外国におけるこれに相当するものを含む。)及びノウハウ(秘匿することが可能な技術情報であって、かつ、財産的価値を有するものをいう。)の利用権は会社が取得する。

\article{業務発明}
従業者等が職務発明等以外の発明等であってその性質上会社の業務範囲に属するもの(以下「業務発明等」という。)を行ったときは、従業者等は、その承継その他業務発明等に係る権利の取扱いについて、第三者に優先して会社と協議しなければならない。
\term 前項の協議により、会社と従業者等とで譲渡の合意が成立した場合には、別途譲渡契約を締結するものとし、これにより当該発明等に係る特許等を受ける権利等が当社に譲渡されるものとする。

\article{権利の処分}
会社は、職務発明等について特許等を受ける権利を取得したときは、当該職務発明等について特許等の出願を行い若しくは行わず、又はその他処分する方法を自由に決定することができる。
\term 出願の有無、取下げ又は放棄、その形態及び内容その他一切の職務発明等の処分については、会社の判断するところによる。

\article{協力義務}
発明者等及び当該職務発明に関与した従業者等は、会社の行う特許等の出願その他特許等を受けるために必要な措置に協力しなければならない。

\article{発明補償}
会社は、職務発明等について特許等を受ける権利を取得したときは、発明者等である従業者等に対し、次の各号に掲げる対価(以下「補償金」という。)を支払うものとする。ただし、発明者が複数あるときは、会社は、各発明者の寄与率に応じて按分した金額を支払う。また、職務発明等が考案である場合には次の各号の額に2分の1を乗じた額を用い,職務発明等が意匠の創作である場合には、次の各号の額に4分の1を乗じた額を用いるものとする。
\begin{enumerate}
    \item 出願補償金:\ref{para:invention_notice}の規定による届出を受け、発明者等と認める者に対して、算定対象となる職務発明等の出願後 金 110,880円 を速やかに支払う。
    \item 登録補償金:算定対象となる職務発明等が、登録された(特許等を取得した)ときは、登録後 金 221,760円 を速やかに支払う。
\end{enumerate}
\label{para:invention_compensation}
\term 前項に定める補償金の算定は、前項の対象となる出願が同一のグループに属する場合には、同一グループ内の最先の出願を対象として行うものとする。なお、本項でいうグループとは、同一のパテントファミリーに属する特許出願、実用新案登録出願及び意匠登録出願のほか、これらの出願に記載された発明、考案又は意匠についての特許出願、実用新案登録出願又は意匠登録出願(分割出願、変更出願、継続出願、一部継続出願、関連意匠出願等のこれらに相当する特許出願、実用新案登録出願及び意匠登録出願を含む)をいう。
\term 本条並びに\ref{para:invention_opposition}及び\ref{para:invention_retiree}の規定については、発明者等が役員である場合には適用しない。発明者等の全部又は一部が役員である場合には、役員である発明者等に対しては第1項に基づく補償金の支払いは行わない。また、共同発明等の場合であって、その発明者等の一部に役員が含まれる場合には、当該役員を含めて\ref{para:invention_collaboration}に基づく分配を行った上で、役員以外の発明者等に対してのみ補償金の支払いを行う。なお、発明者等が役員である場合の補償金その他当該職務発明等に係る相当の利益は、当該役員の報酬等(会社法第361条第1項に規定する「報酬等」を意味する。)を定める際に当該役員から相当の利益の支払申請を受けた場合に、当該役員の報酬等を定めるに際して決定する。

\article{共同発明等}
共同発明等の場合、\ref{para:invention_compensation}に規定する補償金については、発明者等各々の寄与度に応じて分配する。
\label{para:invention_collaboration}
\term 会社は、発明者等の氏名又は発明者等の間の寄与度等について発明者等の間で意見が異なる届出を受けたときは、職務発明等の可否を認定する前に、関係者等から 必要に応じて意見を聴取し、発明者等の氏名又は発明者等の間の寄与度等を決定する。
\term 他社との共同研究により職務発明等が生じたときは、会社は、共同研究の相手方との協議により当該権利の帰属とその持分を定める。この場合において、当該職務発明等に係る権利について共同研究の相手方に持分の帰属を認めたときは、従業者等に帰属する持分の取扱についても、本規程を適用する。

\article{外国における権利の取扱い}
本規程の規定は、外国における本規程が対象とする権利及び同種の権利の取扱いについて準用する。

\article{異議申立て}
会社は、補償金の内容を決定した後これを付与する前に、補償金の付与を受ける従業者等に対し、補償金の対象である職務発明等、補償金の算出方法並びに次項に掲げる異議申立てができること、その提出先及び提出期限を書面により通知しなければならない。
\label{para:invention_opposition}
\term 発明者は、会社から付与された補償金の額に意見があるときは、その通知を受けた日から60日以内に、会社に対して書面により意見の申出を行い、説明を求めることができる。

\article{秘密保持}
職務発明等に関与した従業者等は、職務発明等の内容その他会社の利害に関する事項について、当該事項が公知となるまで、秘密を守らなければならない。
\term 前項の規定は、従業者等が会社を退職した後も適用する。

\article{退職者の発明}
従業者等が会社在職中に完成した職務発明等については、当該職務発明等が完成したことが当該従業者等の退職後に判明した場合であっても、本規程を適用する。

\article{退職者の取扱等}
会社は、従業者等の退職に際し、当該従業者等との合意により、当該従業者等に対する補償金を退職時に一括して付与することができる。 
\label{para:invention_retiree}
\term 従業者等は、当該従業者の退職後においても補償金の付与を受ける場合、\ref{para:invention_opposition}に基づく異議申立権を放棄したものとし、また、会社に対し、自己の住所及び補償金の振込先口座その他補償金の付与に必要な情報を通知しなければならない。当該通知がない場合には、退職日より30日経過後、補償金その他当該職務発明等に係る相当の利益の支払請求権を放棄したものとする。

\article{派遣社員による職務発明}
第三者から会社に派遣されている従業者等による派遣期間中の職務発明の取扱いについては、会社と派遣会社との事前の取決め及び会社と当該従業者等との事前の契約に従うものとする。

\article{秘匿された職務発明}
会社において特許等の出願をしないと決定した特許性を有する職務発明等について、本規程の出願補償金の規定を準用する。この場合、会社が出願しないと決定した時点で出願したとみなして出願補償金の支払いを行う。

\article{準拠法}
本規程は、日本法に準拠し、日本法に従って解釈される。

\article{改定}
本規程の改定は、経済情勢、社会情勢等の変動による労務、経営上の環境変化に応じて見直しを行い、所要の改定を行う。
\term 本規程の改定にあたっては、会社と従業者とが協議を行うものとする。

\subsection*{附則}
\article{施行日}
本規程は、令和3年8月1日から施行する。
\article{遡及適用}
本規程は、施行日前に従業者等によってすでになされている発明等についても適用する。

\clearpage
\section{出張旅費規程}

\article{目的}
この規程は、役員及び従業員(以下「従業員等」という。)が当社の指示命令により、出張する場合の旅費について定めたものである。

\article{出張の定義}
この規則における出張とは、宿泊を必要とする出張(以下「宿泊出張」という。)、もしくは片道80 km以上の移動を伴う出張(以下「日帰り出張」という。)をいう。ただし、会社が必要と認めた場合には、片道80 km に満たない場合であっても、出張として取り扱う場合がある。
\term 片道80 km未満の場所に外出する場合は、単なる外出として交通費実費を支給し、日当は支払わないこととする。
\term この規則における距離計算は、原則として従業員等が通常勤務している場所を始点として計算する。通常勤務している場所とは、週あたりの標準労働時間のうち半分以上の時間を勤務している場所を指し、従業員等の自宅又は本店所在地のいずれかとする。
\term 会社が業務上必要と認めた場合に限り、出張前後途中に前泊、延泊、迂回、途中下車等をする旨を記載した上で出張の申請をし決裁を得た上で実費精算できるものとする。なお、出張前後途中に私用による前泊、延泊、迂回、途中下車等をする場合、それに係る日数は出張の対象日数として数えないものとする。
\term 緊急若しくはその他の事情により、上限額を超えた宿泊先、移動手段しか手配ができない場合は、その旨を記載した上で申請をし決裁を得た上で実費精算できるものとする。

\article{出張の申請}出張をしようとする従業員等は、事前に出張申請を行い、決裁者の決裁を得るものとする。なお、宿泊出張(海外)の場合は、従業員等の安全確認のため、航空機の発着便名及び出張中の宿泊先名を出発日前日までに決裁者に申告するものとする。

\article{出張中の勤務時間の取り扱い}
出張中の勤務時間は、\ref{para:standardworktime}又は\ref{para:discretiontargetworktime}に定める標準労働時間を勤務したものとみなす。ただし、標準労働時間のみでは業務の遂行が困難と会社が認めるときは、この限りでない。

\article{旅費の種類}旅費の種類は、次の各号に定めるとおりとする。
\begin{enumerate}
    \item 日当
    \item 宿泊費
    \item 交通費(鉄道賃、航空賃、車賃等の実費)
    \item その他手数料(展示備品賃借料、展示・配布物配送料、学会参加費、ESTA等のビザ申請手数料、予防接種代、損害保険料等)
    \item 保護責任費用
\end{enumerate}

\article{日当}日当の種類及び金額は次の各号に定めるとおりとする。なお、宿泊出張の日当は出張した日数に応じて(例: 1泊2日の出張の場合、2日分)支給する。
\begin{enumerate}
    \item 日帰り出張:1,500円/日
    \item 宿泊出張(国内):3,000円/日
    \item 宿泊出張(海外):8,000円/日
\end{enumerate}

\article{宿泊費}宿泊費は実費精算とし、上限額は次の各号に定めるとおりとする。
\begin{enumerate}
    \item 国内:12,500円/日
    \item 海外(国・地域の分類は、外務省HP(\url{https://www.mofa.go.jp/mofaj/area/index.html})に則るものとする):
    \begin{enumerate}
        \item A地域 北米、中南米、欧州、中東、アフリカ、オーストラリア、ニュージーランド、香港、シンガポール:30,000円/日
        \item B地域 A地域以外の地域:25,000円/日
    \end{enumerate}
\end{enumerate}
\term 緊急、為替レート変動、もしくはその他事情により、前項で定めた内容及び上限額を超えた宿泊先しか手配できない場合は、その旨を記載した上で申請をし決裁を得た上で実費精算できるものとする。

\article{交通費の計算方法}
交通費は、旅費の計算の基礎となる旅行の起点及び終点を、原則として通常勤務地又は用務地を出発地又は到着地として最も合理的な経路及び方法により算定する。ただし、業務上の必要又は天災その他やむを得ない事情により、これにより難い場合は、現に利用した経路及び方法により算定することができる。
\term 鉄道賃の額は、運賃の額に加えて、急行料金(特別急行料金(指定席特急料金を含む)を含む)を徴する列車の場合は急行料金の額及び座席指定料金を徴する列車の場合は座席指定料金の額を実費で精算できるものとする。特別車両料金は、料金を徴する座席にのみ空席がある場合かつ後続列車の選択が経済的合理性を有しない等の理由がある場合に限って、実費精算できる。また、寝台料金が発生する場合は前条の宿泊費として取り扱う。なお、IC カード乗車券、近距離きっぷ、路線バスその他の乗車の都度に領収書の交付を受けることが困難な公共交通機関を利用した場合は、「出張報告書兼旅費精算書」に出発地(駅)及び到着地(駅)を記載の上、旅費精算を行うものとする。
\term 航空賃の額は、現に支払った運賃によるものとし、エコノミークラス(プレミアムエコノミーを含む)までとする。旅費精算時に、航空賃の支払いを証明するに足る資料(例えば、航空券の半券等)を添付しなければならない。
\term 車賃の額は、バス運賃・タクシー料金は運賃の額を実費で精算できるものとする。レンタカー料金、自家用車のガソリン代については、移動距離に応じた実費額、高速道路・駐車料金はかかった実費額を精算できるものとする。

\article{保護責任費用}
当社の指示命令により臨時に出張するにあたって、従業員等が保護している老年者、幼年者、身体障害者又は病者(以下、「被保護者」という。)に対して生存に必要な保護が困難となるときに、被保護者の保護を受けるための訪問介護、ショートステイサービス、近親者の招致その他の保護責任費用を実費精算で支給する。上限額は次の各号に定めるとおりとする。
\begin{enumerate}
    \item 被保護者一人あたり:7,500円/日
\end{enumerate}

\article{旅費の精算}
旅費の精算については、手続きを経て承認されたものについて、原則として出張終了後、従業員等が自ら「出張報告書兼旅費精算書」にて精算申請を行うものとする。
\term 決裁者によって申請が承認された後、従業員等の給与振込用の口座に支払うこととする。

\article{出張予定の変更、事故・災害時の対応}
出張中において、予定していた日程を変更する事態が生じた場合は、直ちに所属長へ連絡し、承認を得ること。
\term 出張中の者が、負傷、疾病、もしくは災害等により、やむを得ず予定された日程を超えて滞在することとなった場合は、原則としてその後の日当は支給しない。会社が必要と認めた場合に限り、宿泊費は支給するものとする。なお、宿泊費の申請の際は、医師の診断書等、事実を証明できる書類を添付すること。

\subsection*{附則}
\article{施行日} 
本規程の改正は、令和5年12月28日から施行する。

%CERTIFICATION FIELD%
\begin{flushleft}\
\today\\
\vspace{10pt}
上記は当社の現行就業規則と相違ない。\\
\vspace{10pt}
\MakeSignatureField
\end{flushleft}
%CERTIFICATION FIELD%
\end{document}
